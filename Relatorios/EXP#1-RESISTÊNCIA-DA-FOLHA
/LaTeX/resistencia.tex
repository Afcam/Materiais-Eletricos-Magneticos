\section{Respostas}
\subsection{Resistência de Folha das Trilhas}
Para um condutor podemos definir que sua resistência é dada por:
\begin{equation}\label{eq1}
  R = \rho\frac{L}{A}
\end{equation}
onde, $\rho$ é a resistividade, $L$ o comprimento percorrido pela corrente $I$ e $A$ é a área da seção transversal.

Desenvolvendo a equação e substituindo  $A = t.W$, sendo W a largura  e $t$ a expessura da trilha:

$$ R = \rho\frac{L}{A} = \rho\frac{L}{t.W} = R_s\frac{L}{W}  $$
% $$  R_s = \frac{\rho}{t}$$

Agora combinando a resistividade e a expessura $t$, podemos definir que a resistência de folha é: 
\begin{equation}\label{eq2}
  R_s = \frac{\rho}{t}  
\end{equation}
desde que $L$ e $W$ sejam de mesma dimensão.


Cálculos para a Trilha 1:
$$ R_s = R_{T1}\frac{W_{T1}}{L} = 19,00\cdot10^3\cdot\frac{1}{18} $$
$$R_{sT1} = 1,0 \bar{5}K{\frac{\Omega}{\square}} $$

Cálculos para a Trilha 2:
$$ R_s = R_{T2}\frac{W_{T2}}{L} = 7,92\cdot10^3\cdot\frac{2}{18} $$
$$R_{sT2} = 0,88K{\frac{\Omega}{\square}} $$
Desta forma o valor médio da resitência de folha é :
$$R_s \simeq0,965K{\frac{\Omega}{\square}}$$.

\subsection{Resistividade do Grafite}

Dados : $\rho_{grafite} = 7,8\times10^{-6}[\Omega\cdot m]$\\ 
Sendo que espessura da trilha é calculada pela fórmula \ref{eq2}.
$$ t = \frac{\rho}{R_s}  $$

Assim, para a espessura da Trilha 1 temos:

$$t_{T1} = \frac{\rho}{R_{sT1}}=\frac{7,8\times10^{-6}}{1,0\bar{5}\times10^3}  $$
$$=7,3894\times10^{-9}[m]$$

%$$t=7,3894[nm]$$ 

Portanto temos para a trilha 1 \quad $t_{T1}=7,3894nm$.

Para a Trilha 2:

$$t_{T2} = \frac{\rho}{R_{sT2}}=\frac{7,8\times10^{-6}}{0,88\times10^3}  $$
$$=8,8636\times10^{-9}[m]$$

%$$t=7,3894[nm]$$ 

Portanto temos para a trilha 2 \quad  $t_{T2}=8,8636nm$.

Desta forma o médio da espessura da trilha é:
$$t=8,1265nm$$.

%\newpage
\subsection{Resistência Interna do Multímetro}
A resistência interna do multímetro\, \textit{TOZZ-DT830D} é de: $$R_i=6,6\Omega$$
As medidas são afetadas por causa da sua resistência não nula, desta forma diversos efeitos como o de dissipação de potência, tempo de acomodamento e quedas de tensões podem ocorrer; Afetando a precisão das medidas.


\subsection{Precisão e Acurácia do Mulda trilhatímetro}
A precisão e acurácia do multímetro variam conforme a escala de medição, e é definida pelo fabricante.
No experimento utilizamos o modelo \textit{TOZZ-DT830D}


Modo Ohmímetro
\begin{center}
\begin{tabular}{lll}
Escala & Precisão & Acurácia\\
200$\Omega$ & 0,1$\Omega$ &  $\pm$1,0$\pm$2D\\
2000$\Omega$ & 1$\Omega$ &  $\pm$0,8$\pm$2D\\
20k$\Omega$ & 10$\Omega$ &  $\pm$0,8$\pm$2D\\
200k$\Omega$ & 100$\Omega$ &  $\pm$0,8$\pm$2D\\
2000k$\Omega$ & 1k$\Omega$ &  $\pm$1,0$\pm$2D\\
\end{tabular}
\end{center}

Modo Voltímetro
\begin{center}
\begin{tabular}{lll}
Escala & Precisão & Acurácia\\
200mV & 100$\mu$V &  $\pm$0,5$\pm$2D\\
2000mV & 1mV &  $\pm$0,5$\pm$2D\\
20V & 10mV &  $\pm$0,5$\pm$2D\\
200V & 100mV &  $\pm$0,5$\pm$2D\\
1000V & 1V &  $\pm$0,8$\pm$2D\\
\end{tabular}
\end{center}
